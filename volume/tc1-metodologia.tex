Este trabalho possui um forte componente experimental e exploratório. Assim, em termos metodológicos, a cada experiência realizada, os resultados obtidos foram analisados quantitativa e qualitativamente, para orientar as correções nos parâmetros do \emph{parser} ou indicar a necessidade de alterações como: a) de pré-processamento ou pós-processamento dos casos; ou b) no código do \emph{parser}. Nesse sentido, a avaliação quantitativa é um componente importante e será feita de forma rigorosa. Pretende-se utilizar os métodos de avaliação tradicionais de \emph{precision/recall} \cite{black91}.

Para trabalhar com o corpus no formato de entrada para o parser de Bikel foi necessário pré-processamento para eliminar ruídos e criar dados de entrada no formato PTB, mesmos obstáculos encontrados por Baldridge \cite{baldridge06} e Bonfante\cite{bonfante03} em seus trabalhos. 

O Corpus de treino, desenvolvimento e teste utilizado é o Bosque da Floresta Sintática que contém um total de 5221 sentenças separadas em 4177 para treino 522 para desenvolvimento e 522 para teste, nas primeiras baterias de teste utilizamos pouco menos de 100 sentenças para ajustar parâmetros mais rapidamente, na última foi utilizado 520 sentenças na fase de desenvolvimento e teste.

A presença de ruído nos dados do corpus é inevitável pois a maioria das sentenças são anotadas inicialmente de maneira automática, e mesmo após revisão manual algumas inconsistências ainda foram encontradas no decorrer do trabalho. Ruídos são possívelmente provenientes da complexidade na construções da língua que pode gerar ambiguidade ou estruturas complexas, o que dificulta a análise (em alguns casos dificulta até a análise humana). 

%Não foi feita alteração quanto as TAGS utilizadas no corpus de trabalho, apenas quanto a granularidade das TAGS, agrupando em uma unica TAG quando houver sub grupo da mesma, para avaliar a necessidade de sintaticamente diferenciar paralvras com funções diferentes a pontuação de hífen, que será substituida por Underline.

O Conjunto de experimentos foi dividido em subgrupos que tentam avaliar a melhor configuração com respeito a algum fator relevante quanto a utilização de subcategorias das TAGS, regras para determinar o núcleo das sentenças e parâmetro de utilização da ferramenta. Para cada grupo é eleita uma melhor configuração e a partir destes os prosseguem.

Para agilizar este processo foi desenvolvido um ambiente de testes em que os experimentos são programados e automatizados. Em particular os aspectos de pré-processamento e criação das seleções do corpus quanto a filtros necessários para diferentes experimentos.



