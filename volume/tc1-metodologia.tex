Este trabalho possui um forte componente experimental e exploratório. Assim, em termos metodológicos, a cada experiência realizada, os resultados obtidos devem ser analisados quantitativa e qualitativamente, para orientar as correções nos parâmetros do \emph{parser} ou indicar a necessidade de alterações como: a) de pré-processamento ou pós-processamento dos casos; ou b) no código do \emph{parser}. Nesse sentido, a avaliação quantitativa é um componente importante e será feita de forma rigorosa. Pretende-se utilizar os métodos de avaliação tradicionais de \emph{precision/recall} \cite{black93} e possivelmente outras a definir.

Será usado um sistema de controle de versão que permite que se trabalhe com diversas versões dos arquivos de trabalho
e  versões do software durante nossos testes e implementações.

Inicialmente trabalhamos sobre a influêcia da escolha do núcleo \emph{(Head)} dos constituíntes que é o principal fator na implementação dos modelos propostos por Collins \cite{collins99}.

O corpus do Bosque ja tem anotados muitos dos \emph{heads} das sentenças disponibilizadas, porém o parser de bikel precisa de todos os \emph{heads}, e verificou-se que nem todas as sentenças possuem essa informação, tendo que ser analisada e construída de forma empírica. 

A ferramenta de Bikel tem um módulo para especificação de regras para determinar o \emph{head} de um constituinte e optou-se por se usar esse módulo e construir regras baseadas nas regras de formação da lingua portuguesa.

O primeiro usou-se as regras default para o ingles para treino e parser da ferramenta para avaliarmos o percentual de acerto quando se utiliza regras para o ingles aplicados a lingua portuguesa.

Como esperado os resultados foram bem baixos, uma vez que as regras de formação de sentenças validas para o inglês e português nao seguem os mesmos padrões quanto a núcleo das sentenças. 

Nos experimentos seguintes reduzimos o número de regras apenas para as regras básicas de encontro de heads de sentenças, testou-se primeiro selecionando como head de um constituinte o a palavra mais a direita e logo a palavra mais a esquerda. 

Os resultados deste teste mostrou que o conjunto de regras ....

è interessante mensionar que a construção das regras foi incremental  ... a título de curiosidade  figura tal ...

O primeiro conjunto de regras usadas é o que estamos usando atualmente bem melhor que o original, conforme resultados apresentados, esses resultados foram ... na arvore de experimentos


Formado o conjunto de regras de head-find . A segunda sequencia de experimentos tera a ver com a granilaridade das POS tags. ....

...


A ideia basica é que tags devem ser distintos quando a categorias tem distribuições sintáticas diferentes, por outro lado se duas classes tem mesma distribuição ...,  ...

Verificou-se conforme esperado que para verbos é bastante relevante as sub-categorias pois a distribuição sintática das diferentes categorias verbais é bem distinta, da mesma forma para pronomes porque os possessivos tem diferente distribuição que os pessoais; os .. tem posição de pré-modificadores nominal e os outros ...

O último caso relativo a nomes refere-se  

 

Detalhar mais o metodo ......
