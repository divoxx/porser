As informações sobre o projeto Semantic Share são baseadas na página do projeto (\url{http://semanticshare.di.fc.ul.pt/}), na página do projeto submetido ao Ministério de Ciência e Tecnologia (MCT) de Portugal (\url{http://www.fct.mctes.pt/projectos/pub/2006/Painel_Result/vglobal_projecto.asp?idProjecto=81157&idElemConcurso=896}) e em dados extraídos de uma das materializações de uma parte inicial do corpus que temos disponível.

\section{Objetivos principais} % (fold)
\label{sec:semantic_objetivos}

Um objetivo principal do SemanticShare é o desenvolvimento para o português de corpora anotados da mais recente geração e da próxima geração - um PropBank e um LogicalFormBank -, dos quais uma parte é paralela a bancos de dados similares que estão a ser produzidos para outros idiomas, em outros projetos.

\section{Características principais} % (fold)
\label{sec:semantic_catacteristicas}


Estes corpora são diferentes materializações de um banco único de enunciados e correspondentes representações gramaticais, com as seguintes características principais:


\subsection{Abrangência} % (fold)
\label{sub:semantic_abrangencia}

Contêm informação morfológica, sintática e semântica integrada.


\subsection{Acessibilidade} % (fold)
\label{sub:semantic_acessibilidade}

Podem ser apresentadas em uma ou mais de entre várias vistas:

\begin{enumerate}
  \item Frases
  \item Segmentos lexicais
  \item Lemas
  \item Traços de flexão
  \item Etiquetas morfossintáticas
  \item Entidades nomeadas e unidade multi-palavra
  \item Árvores de constituintes
  \item Árvores de funções e papéis semânticos
  \item Formas lógicas
\end{enumerate}



\subsection{Precisão} % (fold)
\label{sub:semantic_precisao}
Cada representação é arquivada por seleção humana, depois de ter sido gerada por uma analisador gramatical;


\subsection{Profundidade} % (fold)
\label{sub:semantic_profundidade}

São arquivadas num formato de representação interno que é linguisticamente bem informado, seguindo um quadro gramatical de primeira linha para a lingüística computacional (HPSG);


\subsection{Evolução} % (fold)
\label{sub:semantic_evolucao}

São apoiadas por ferramentas de desenvolvimento de corpora avançadas que asseguram uma extensão fácil das estruturas anotadas quando mais informação de mais dimensões lingüísticas possa ter de ser adicionada em extensões futuras (e.g. tempo, resolução de anáfora, etc), ou quando a cobertura da gramática seja aprofundada.


\section{Infra-estrutura de investigação} % (fold)
\label{sec:semantic_infra}

Estes objetivos estão ao alcance do projeto na medida em que tiram partido da maioria das ferramentas e recursos desenvolvidos pela equipa do SemanticShare em projetos anteriores bem sucedidos, nomeadamente o projeto TagShare, do qual o SemanticShare é uma continuação. Eles constituem uma coleção única de ferramentas de última geração para o Português -- segmentador de frases e lexemas, etiquetador morfossintático, lematizador, analisador morfológico, reconhecedor de entidades nomeadas --, juntamente com o respectivo corpus de 1 milhão de ocorrências, anotado com precisão de acordo com as dimensões 1.-6. acima (serviços online em http://nlxgroup.di.fc.ul.pt).


\section{Apoio de uma comunidade internacional de investigadores} % (fold)
\label{sec:semantic_apoio}

Tudo isto será realizado com o apoio do consórcio DELPH-IN, uma iniciativa de nível mundial que visa dinamizar investigação de ponta em processamento lingüístico profundo através da partilha de ferramentas de desenvolvimento "open source", de recursos e de boas práticas (http://wiki.delph-in.net) entre os seus participantes convidados (vd. carta de convite em http://www.di.fc.ul.pt/~ahb/semanticshare.htm).

A sua plataforma tecnológica e a sua ferramenta de anotação sem rivais permitem avanços rápidos na concretização dos objetivos do projeto, dando assim continuidade à cooperação anterior, nomeadamente no quadro do projeto GramaXing, em que uma gramática para o processamento lingüístico profundo foi desenvolvida e está a ser mantida.

Para além disso, parte do banco lingüístico a ser desenvolvido é a componente portuguesa de bancos paralelos que estão a ser desenvolvidos para outros idiomas por outros membros do DELPH-IN, segundo requisitos similares.


Estes corpora anotados representam recursos chave para o processamento do Português, incluindo:


\begin{itemize}
  \item fornecer uma base empírica para o estudo lingüístico deste idioma e para o desenvolvimento de ferramentas elaboradas manualmente;
  \item treinar ferramentas de base estatística para o processamento superficial e profundo, incluindo parsers, etiquetadores de papéis semânticos, etc;
  \item avaliar ferramentas de processamento;
  \item  apoiar a experimentação de abordagens inovadoras em PLN multilingue, incluindo tradução automática estatística ou meta-anotação automática para a web semântica, etc...
\end{itemize}



\subsection{Esquema de Anotação do Projeto Semantic Share} % (fold)
\label{sub:semantic_anotacao}

O esquema de anotação do projeto Semantic Share são visões baseadas na anotação base do projeto que utiliza HPSG \cite{branco08}.

A partir desta anotação são extraídas ``visões'' no formato de árvores de constituintes. Estas árvores formam o corpus de pesquisa utilizado neste trabalho.


A tabela \ref{tbl:semantic_share_pos} mostra os rótulos de POS e a tabela \ref{tbl:semantic_share_cats}, os rótulos sintáticos.

\begin{table}
   \centering
   \small
   %%\setlength{\arrayrulewidth}{2\arrayrulewidth}
   %%\setlength{\belowcaptionskip}{10pt}
   \caption{\it Tags de Part-of-Speech do projeto Semantic Share.}

    \begin{tabular}{ | p{3cm} | p{10cm} | }
      \hline
        \textbf{Símbolo} & \textbf{Categoria}\\
        \hline
        \hline

    A&Adjetivo\\
    \hline
    ADV&Adverbio\\
    \hline
    C&Complementador ( que)\\
    \hline
    CARD&Cardinal\\
    \hline
    CONJ&Conjunção\\
    \hline
    D&Determinador\\
    \hline
    DEM&Pronome demonstrativo\\
    \hline
    N&Nome\\
    \hline
    P&Preposição\\
    \hline
    PNT&Símbolo de pontuação\\
    \hline
    POSS&Pronome possessivo\\
    \hline
    PPA&Particípio passado\\
    \hline
    QNT& Quantificador\\
    \hline
    V& Verbo\\
    \hline


   \end{tabular}
   \label{tbl:semantic_share_pos}
\end{table}


\begin{table}
   \centering
   \small
   %%\setlength{\arrayrulewidth}{2\arrayrulewidth}
   %%\setlength{\belowcaptionskip}{10pt}
   \caption{\it Tags sintáticos do projeto Semantic Share.}

    \begin{tabular}{ | p{3cm} | p{10cm} | }
      \hline
        \textbf{Símbolo} & \textbf{Categoria}\\
        \hline
        \hline

        ADVP& Sintagma adverbial \\
        \hline
        AP& Sintagma Adjetival\\
        \hline
        CONJP&Sintagma coordenativo\\
        \hline
        CP&Sintagma Complementizador\\
        \hline
        NP&Sintagma nominal\\
        \hline
        N'&Projeção intermediária entre N e NP\\
        \hline
        pp&Sintagma preposicional\\
        \hline
        PPA'&Projeção intermediária entre PPA e PPAP\\
        \hline
        PPAP&Sintagma de oração Passiva\\
        \hline
        S&Sintagma de sentença\\
        \hline
        SNS&Sintagma de sentença sem sujeito\\
        \hline
        VP&Sintagma verbal\\
        \hline

   \end{tabular}
   \label{tbl:semantic_share_cats}
\end{table}

Nota: Alguns sintagma são com informação de extração, por exemplo ``S/NP'' significa sintágma de sentença com extração de NP (sujeito), VP/NP significa sintágma verbal com extração de NP (objeto), e assim por diante. As ocorrências desse tipo encontradas no corpus foram essas: S/ADVP, S/AP, S/PP, SNS/ADVP, SNS/NP, VP/ADVP, VP/AP, VP/PP.

