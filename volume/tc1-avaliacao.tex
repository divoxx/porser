As medidas de avaliação do \emph{parser} seguirão a proposta de GEIC/Parseval \cite{black91}, possivelmente adaptado conforme \cite{collins97} para ignorar pontuação e não considerar a marcação de POS na avaliação. 

Em particular, serão usadas as medidas de \emph{Labeled Precision} (LP) e \emph{Labeled Recall} (LR) e sua média harmônica ($F_{\beta=1}$) ou \emph{F-Score}, descritas abaixo:
\\
$$LP = \frac{n\acute{u}mero\; de\; constituintes\; corretas\; na\; an\acute{a}lise\; proposta}{n\acute{u}mero\; de\; constituintes\; da\; an\acute{a}lise\; proposta}$$
\\
$$LR = \frac{n\acute{u}mero\; de\; constituintes\; corretas\; na\; an\acute{a}lise\; proposta}{n\acute{u}mero\; de\; constituintes\; do\; \mathit{treebank}\; analisado}$$
\\
$$F_{\beta=1} = \frac{2*LP*LR}{LP+LR}$$
\\
O termo \emph{Labeled} se refere ao fato de que uma constituinte, para contar como corretamente recuperado, deve acertar a extensão correta do texto bem como o rótulo do constituinte.

O procedimento de avaliação compara a saída do parser com as análises anotadas no \emph{treebank}; usa a informação de parentização da representação do \emph{treebank} de uma sentença e a análise produzida pra computar tres medidas: \emph{crossing braqkets, precision e recall}, neste trabalho não utilizaremos a medida de \emph{crossing braquets}.

Estas métricas são chamadas métricas estruturais, e são baseadas na avaliação dos limites dos sintágmas. Os algoritmos de parsing tem por objetivo otimizar uma métrica em comum, que é a probabilidade de se ter uma árvore corretamente rotulada, ou seja, com uma marcação correta dos limites dos constituíntes. Assim dado um nó em uma árvore sintática, a sequencia de palavras dominadas por esse nó forma um sintágma, sendo o limite do sintágma representado por um interválo inteiro \emph{[i,j]}, em que \emph{i} representa o índice da primeira palavra e \emph{j} o da última palavra do sintágma.

Black \cite{black91} propõe três medidas estruturais para avaliar sistemas de parsing: Labeled Precision, Labeled Recall e Crossing-Brackets. Segundo Lin (1995), esse esquema de avaliação pode ser classificado como em nível de sintágma, ou nível de sentença. 
As medidas de Labeled Precision e Labeled Recall são computadas da seguinte forma:

Os limites dos sintágmas na resposta (análise produzida pelo parser) e no gold (análise do treebank) são tratados como dois conjuntos (A e K), em que A é a análise obtida do parser proposto e K, o gold do treebank a ser usado na avaliação. O Labeled Recall
é definido como a percentagem no gold que também é encontrada na resposta ((A K)/K). A Labeled Precision é definica como a percentagem de limites no sintagma da resposta que também é encontrada no gold ((A K)/A).

As medidas propostas no PARSEVAL partem de um pressuposto de que um constituinte esta correto se corresponde ao mesmo conjunto de palavras (ignorando qualquer caractere de pontuação) e tem o mesmo rótulo que um constituínte no treebank.


Exemplo: Considere (1) gold e (2) análise do parser:\\ \\

1. (S (ACL (NP (ART Um) (N arquiteto)) (PP (PRP para) (NP (NUM 800) (M Km2)))))
2. (S (NP (ART Um) (N arquiteto) (PP (PRP para) (NP (NUM 800) (M Km2)))))

Temos:\\ \\
Limite para os sintagmas em (1): (S 0,4); (ACL 0,4); (NP 0,1); (PP 2,4); (NP 3,4). \\
Limite para os sintagmas em (2): (S 0,4); (NP 0,1); (PP 2,4); (NP 3,4).

Pontuações em (2): Labeled Precision = 4/4 = 100{\%}, Labeled Recall = 4/5 = 80{\%}




