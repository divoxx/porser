A motivação pode ser dividida em dois aspectos principais: científico e tecnológico.

A motivação científica é a obtenção do conhecimento e o melhor entendimento a respeito de como as linguagens funcionam. Nenhuma das disciplinas tradicionais possui, isoladamente, ferramentas necessárias para decifrar completamente a produção e a compreensão linguísticas que os seres humanos possuem. Porém, é possível utilizar programas de computadores para implementar essa complexa teoria, de modo que seja possível testá-la, verificá-la e incrementalmente melhorá-la. Ao aprofundar o estudo deste processo, podemos desenvolver um entendimento a respeito de como os seres humanos processam as línguas.

Quanto à natureza tecnológica, a maior parte do conhecimento humano está armazenada de forma linguística, escrita ou falada, e computadores que conseguissem ``entender'' linguagem natural poderiam acessar toda essa informação. Outro aspecto relevante é a possibilidade de melhorar a interação humano-computador, aumentando o nível de acessibilidade, o que tornaria mais simples a utilização de ferramentas computacionais por pessoas com necessidades especiais.

Atualmente, a evolução do poder computacional e a construção de grandes \emph{treebanks} possibilitam a utilização de técnicas mais avançadas, que utilizam grande quantidade de informação e processamento para tentar resolver esses problemas. Técnicas como \emph{parsing} probabilístico, que utiliza técnicas de aprendizado e cálculos estatísticos baseados em um banco de dados manualmente anotado - conhecido como \emph{corpus} ou \emph{treebank} - para identificar as informações sintáticas corretas, têm se mostrado bastante eficazes, na comparação com outros métodos, e suficientemente satisfatórias, por conta dessa evolução computacional.

Muitas pesquisas e trabalhos vêm sendo realizados, com foco em vários idiomas, notadamente o inglês \cite{prolo03,charniak97,collins97}, entretanto verifica-se uma carência de pesquisas, ferramentas, recursos linguísticos e humanos para tratar computacionalmente a língua portuguesa.
Existem alguns trabalhos \cite{baldridge06,bick00,bonfante03} mas é fato reconhecido pelos pesquisadores que ainda não se atingiu um resultado de nível desejável.

Michael Collins, no final da década de 1990, desenvolveu três modelos de \emph{parsing} probabilístico, sendo os últimos extensões aos anteriores. Estes modelos e o seu \emph{parser} são até hoje referência na área e continuam sendo utilizados. Posteriormente, em 2004, Dan Bikel reimplementou o \emph{parser} de Collins, tornando-o mais parametrizável e extensível. Ambos os \emph{parsers} foram amplamente testados para a língua inglesa e, na atualidade, as tentativas de se construir um \emph{parser} probabilístico para a língua portuguesa não têm sido, até o momento, satisfatórias.

Finalmente, outro motivador crucial para este trabalho de conclusão é o trabalho em desenvolvimento pelo projeto Semantic Share da Universidade de Lisboa, quanto ao desenvolvimento de um \emph{corpus} para a língua portuguesa melhor anotado. Este \emph{corpus} contém dados linguísticos de fala ou escrita, servindo de base de uso para o \emph{parser} estatístico utilizado.




